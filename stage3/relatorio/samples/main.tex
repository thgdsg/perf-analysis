%%
%% This is file `sample-acmsmall.tex',
%% generated with the docstrip utility.
%%
%% The original source files were:
%%
%% samples.dtx  (with options: `all,journal,bibtex,acmsmall')
%% 
%% IMPORTANT NOTICE:
%% 
%% For the copyright see the source file.
%% 
%% Any modified versions of this file must be renamed
%% with new filenames distinct from sample-acmsmall.tex.
%% 
%% For distribution of the original source see the terms
%% for copying and modification in the file samples.dtx.
%% 
%% This generated file may be distributed as long as the
%% original source files, as listed above, are part of the
%% same distribution. (The sources need not necessarily be
%% in the same archive or directory.)
%%
%%
%% Commands for TeXCount
%TC:macro \cite [option:text,text]
%TC:macro \citep [option:text,text]
%TC:macro \citet [option:text,text]
%TC:envir table 0 1
%TC:envir table* 0 1
%TC:envir tabular [ignore] word
%TC:envir displaymath 0 word
%TC:envir math 0 word
%TC:envir comment 0 0
%%
%% The first command in your LaTeX source must be the \documentclass
%% command.
%%
%% For submission and review of your manuscript please change the
%% command to \documentclass[manuscript, screen, review]{acmart}.
%%
%% When submitting camera ready or to TAPS, please change the command
%% to \documentclass[sigconf]{acmart} or whichever template is required
%% for your publication.
%%
%%
\PassOptionsToPackage{table,dvipsnames}{xcolor}
\documentclass[acmsmall]{acmart}
\usepackage[utf8]{inputenc}
\usepackage[inline]{enumitem}
\usepackage{multirow}
\usepackage{booktabs}
\usepackage{geometry}
\usepackage{graphicx}
\usepackage{subcaption}
\usepackage{placeins}
%%
%% \BibTeX command to typeset BibTeX logo in the docs
\AtBeginDocument{%
  \providecommand\BibTeX{{%
    Bib\TeX}}}
\settopmatter{printacmref=false}
%% Rights management information.  This information is sent to you
%% when you complete the rights form.  These commands have SAMPLE
%% values in them; it is your responsibility as an author to replace
%% the commands and values with those provided to you when you
%% complete the rights form.
\setcopyright{CC}
\copyrightyear{2025}
\acmYear{2025}
\acmDOI{}

%%
%% These commands are for a JOURNAL article.
%%\acmJournal{JACM}
%\acmVolume{37}
%\acmNumber{4}
%\acmArticle{111}
%\acmMonth{8}

%%
%% Submission ID.
%% Use this when submitting an article to a sponsored event. You'll
%% receive a unique submission ID from the organizers
%% of the event, and this ID should be used as the parameter to this command.
%%\acmSubmissionID{123-A56-BU3}

%%
%% For managing citations, it is recommended to use bibliography
%% files in BibTeX format.
%%
%% You can then either use BibTeX with the ACM-Reference-Format style,
%% or BibLaTeX with the acmnumeric or acmauthoryear sytles, that include
%% support for advanced citation of software artefact from the
%% biblatex-software package, also separately available on CTAN.
%%
%% Look at the sample-*-biblatex.tex files for templates showcasing
%% the biblatex styles.
%%

%%
%% The majority of ACM publications use numbered citations and
%% references.  The command \citestyle{authoryear} switches to the
%% "author year" style.
%%
%% If you are preparing content for an event
%% sponsored by ACM SIGGRAPH, you must use the "author year" style of
%% citations and references.
%% Uncommenting
%% the next command will enable that style.
%%\citestyle{acmauthoryear}


%%
%% end of the preamble, start of the body of the document source.
\begin{document}

%%
%% The "title" command has an optional parameter,
%% allowing the author to define a "short title" to be used in page headers.
\title{Avaliação de desempenho do algoritmo PageRank sequencial e paralelo}

%%
%% The "author" command and its associated commands are used to define
%% the authors and their affiliations.
%% Of note is the shared affiliation of the first two authors, and the
%% "authornote" and "authornotemark" commands
%% used to denote shared contribution to the research.
\author{Thiago dos Santos Gonçalves}
\authornote{Ambos os autores contribuíram igualmente neste trabalho.}
\email{tsgoncalves@inf.ufrgs.br}
\affiliation{%
  \institution{Instituto de Informática - UFRGS}
  \city{Porto Alegre}
  \state{Rio Grande do Sul}
  \country{Brasil}
}
\author{João Vitor do Amaral Spolavore}
\authornotemark[1]
\email{joao.spolavore@inf.ufrgs.br}
\affiliation{%
  \institution{Instituto de Informática - UFRGS}
  \city{Porto Alegre}
  \state{Rio Grande do Sul}
  \country{Brasil}
}

%%
%% By default, the full list of authors will be used in the page
%% headers. Often, this list is too long, and will overlap
%% other information printed in the page headers. This command allows
%% the author to define a more concise list
%% of authors' names for this purpose.
\renewcommand{\shortauthors}{Thiago e João}

%%
%% The abstract is a short summary of the work to be presented in the
%% article.
\begin{abstract}
  Este trabalho apresenta uma avaliação de desempenho do algoritmo PageRank, utilizado por muito tempo pelo indexador do \textit{Google}, testado com execução sequencial e paralela. Para isso, esse estudo analisa o comportamento do algoritmo sob diferentes configurações de execução em arquiteturas paralelas. A metodologia experimental consistiu na execução do algoritmo com 9 variações de contagem de \textit{threads}, utilizando 7 grafos de entrada distintos, alternando o estado do \textit{Hyper-Threading}, aplicando diferentes políticas de vinculação de \textit{threads}, e fazendo diferentes análises com o \textit{Intel Vtune Profiler}. Os resultados obtidos demonstram o impacto dessas configurações no tempo de execução e tentam correlacionar o tempo de execução com algumas métricas coletadas e com diferentes variações de configurações. Observamos valores de \textit{speedup} de até $15\times$ dependendo da estratégia de paralelismo adotada para execução.
\end{abstract}

\keywords{PageRank, Google, Computação Paralela, Computação de Alto Desempenho, Algoritmos, Grafos, Avaliação de Desempenho}


%%
%% This command processes the author and affiliation and title
%% information and builds the first part of the formatted document.
\maketitle

\section{Introdução}
\subsection{Descrição do Objeto de Pesquisa}
O algoritmo PageRank, originalmente proposto por Sergey Brin e Larry Page para o motor de busca do \textit{Google}\cite{page1999pagerank}, é um método de análise de \textit{links} que atribui pesos numéricos a cada elemento de uma coleção de documentos ligados, com o objetivo de medir a sua importância relativa. A internet seria modelada como um grafo direcionado $G = (V, E)$, onde os vértices $V$ representam as páginas e as arestas $E$ representam os \textit{hyperlinks}. O conceito central do algoritmo se baseia na ideia de que uma página é importante se for citada por outras páginas importantes.

Matematicamente falando, o cálculo do valor do PageRank ($PR$) de uma página $u$ é definido de forma recursiva. A equação que define a distribuição de probabilidade do "surfista aleatório" (\textit{random surfer}) é dada por:

\begin{equation} \label{eq:pagerank}
    PR(u) = \frac{1-d}{N} + d \sum_{v \in B_u} \frac{PR(v)}{L(v)}
\end{equation}

\noindent onde $d$ é o fator de amortecimento, $N$ é o número total de páginas, $B_u$ é o conjunto de páginas que apontam para $u$, e $L(v)$ é o número de links saindo da página $v$.

Do ponto de vista computacional, que é o foco deste trabalho, o cálculo do PageRank é resolvido através do "Método das Potências"\cite{wiki:Power_iteration}. Este processo envolve a multiplicação iterativa de um vetor de classificação por uma matriz que representa a estrutura do grafo até que os valores entrem em convergência dentro de uma margem de erro predefinida. Devido o enorme volume de dados da matriz esparsa da internet, este algoritmo demanda intenso processamento de ponto flutuante e acesso à memória dependendo da entrada. Essas características fazem com que o PageRank seja um bom algoritmo para estudo de escalabilidade paralela, justificando nosso estudo do comportamento da escolha de \textit{threads}, afinidade e \textit{Hyper-Threading} para otimizar o tempo de convergência.

\subsection{Objetivos do Trabalho}
Com isto definido, nós iremos analisar o desempenho do algoritmo PageRank em uma máquina do Parque Computacional de Alto Desempenho (PCAD) do Instituto de Informática (INF) da Universidade Federal do Rio Grande do Sul (UFRGS) \cite{PCAD}. Para realizar essa análise, utilizaremos o \textit{Intel VTune Profiler} \cite{IntelProfiler}, para obter métricas mais aprofundadas de uso dos núcleos durante a execução do algoritmo, como também executaremos sem esse \textit{profiler} pra obtermos o tempo de execução e calcularmos dados com base nele.

Mais especificamente, dividiremos nosso objetivo principal de analisar o algoritmo PageRank nesses sub-objetivos:
\begin{itemize}
    \item Comparar tempos de execução/\textit{speedup} entre diferentes configurações.
    \item Associar as métricas adquiridas com possíveis causas pra diferenças nos tempos de execução.
\end{itemize}

Assim conseguiremos obter mais informações sobre a escalabilidade do algoritmo PageRank e também descobrir o que limita o desempenho do nosso algoritmo dentro do sistema em que ele foi executado.

\subsection{Revisão da Proposta e Metodologia}
De início, nós planejamos apenas utilizar o \textit{Intel Vtune Profiler} pra obter tanto o tempo de execução quanto as métricas que iríamos utilizar nesse trabalho.
Porém, conforme fomos realizando os experimentos, percebemos que executar apenas com o \textit{Intel Vtune Profiler} não nos entregaria os verdadeiros resultados de tempo devido ao \textit{overhead} introduzido.
Isso nos levou a realizar mais execuções sem esse \textit{profiler}, realizando mais repetições por configuração pra obter maior precisão em nossos resultados de tempo de execução e dividindo nossa análise entre as execuções com o \textit{profiler} e as execuções sem.

Fora isso, desde o início tínhamos confiança de que nosso projeto seria plausível, apenas não sabíamos exatamente como realizar essa análise do desempenho. Com a ajuda do professor, essa avaliação se tornou possível e seguiu critérios de maior rigor científico.


\section{Fundamentação Teórica}
O \textit{Intel VTune Profiler} \cite{IntelProfiler} é uma ferramenta pra análise de desempenho que ajuda os desenvolvedores e analistas de \textit{hardware} a otimizar o desempenho de aplicações e sistemas, identificando gargalos em \textit{softwares} que utilizam CPU, GPU ou FPGA. Ele analisa como uma aplicação usa os recursos de \textit{hardware}, mostrando onde o código está mais lento, identificando problemas na memória \textit{cache}, sincronização de \textit{threads} e eficiência na utilização do processador. A ferramenta fornece informações detalhadas e ajuda a encontrar áreas para melhoria de desempenho em códigos seriais e paralelos. 

O repositório \textit{GAPBS} \cite{beamer2017gapbenchmarksuite} é uma coleção de \textit{benchmarks} de algoritmos de operações em grafos consolidada na comunidade acadêmica, citado em ao menos 20 artigos publicados em revistas e conferências revisadas por pares. Ele foi criado por estudantes da universidade de \textit{Berkley} e possui algoritmos como busca por profundidade, contagem de triângulos, e o nosso objeto de pesquisa, o algoritmo PageRank.
Esse repositório implementa seus \textit{kernels} em C++ e OpenMP \cite{wiki:OpenMP}, com saídas e entradas simples pro usuário final.

O repositório \textit{Stanford Large Network Dataset Collection} (SNAP) \cite{leskovec2016snap} é uma coleção de grafos de entrada da universidade de \textit{Stanford}. Também consolidado academicamente e com vários artigos já publicados \cite{SNAP}, os grafos disponibilizados que representam relações entre páginas da internet ou entre usuários de aplicativos foram essenciais para nosso trabalho.


\section{Metodologia}
Nós consideramos os seguintes grafos de entrada com as seguintes características:

\begin{table}[hbt!]
    \centering
    \caption{Grafos utilizados nos experimentos}
    \label{tab:grafos_entrada}
    
    \begin{tabular}{lrr}
        \toprule
        \textbf{Grafo} & \textbf{Vértices} & \textbf{Arestas} \\
        \midrule
        Friendster     & 65.608.366        & 1.806.067.135    \\
        LiveJournal    & 3.997.962         & 34.681.189       \\
        Orkut          & 3.072.441         & 117.185.083      \\
        BerkStan       & 685.230           & 7.600.595        \\
        Google         & 875.713           & 5.105.039        \\
        NotreDame      & 325.729           & 1.497.134        \\
        Stanford       & 281.903           & 2.312.497        \\
        \bottomrule
    \end{tabular}
\end{table}

Utilizamos a implementação do algoritmo PageRank do repositório \textit{GAPBS}, já mencionado anteriormente. Nesta implementação, ele utiliza uma margem de erro de $1^{-4}$, e um fator de amortecimento de $0,85$.

Executamos nosso algoritmo na máquina blaise do GPPD. Essa máquina possui 2 processadores \textit{Intel(R) Xeon(R) E5-2699 v4} com $256GB$ de memória RAM DDR4, totalizando 88 \textit{threads} e 44 núcleos operando a 2,2$GHz$ e com suporte a \textit{Turbo Boost}, levando os núcleos a operarem a até 3,6$GHz$ com 55$MB$ de memória \textit{cache} compartilhada. Então, para nossos experimentos, consideramos apenas 44 \textit{threads} e 22 núcleos. Os experimentos foram executados no \textit{kernel} Linux versão 4.19.0-25, com a versão do gcc 8.3.0, e as métricas foram coletadas com o \textit{Intel Vtune Profiler} versão 2021.1.

Essas foram as configurações consideradas em nossos testes:

\noindent \textbf{Quantidade de threads:}
\begin{itemize*}[label={}]
    \item 1, \item 4, \item 12, \item 22, \item 28, \item 36, \item 44, \item 66 e \item 88 \textit{threads}.
\end{itemize*}

\vspace{0.5em}

\noindent \textbf{Tipos de análise do \textit{Intel VTune Profiler}:}
\begin{itemize*}[label={}]
    \item hpc-performance;
    \item hotspots;
    \item performance-snapshot;
    \item nenhuma.
\end{itemize*}

\vspace{0.5em}

\noindent \textbf{Políticas de vinculação de \textit{threads}:}
\begin{itemize*}[label={}]
    \item \textit{close};
    \item \textit{spread}.
\end{itemize*}

\vspace{0.5em}

\noindent \textbf{Estado do \textit{Hyperthreading}:}
\begin{itemize*}[label={}]
    \item ligado (\textbf{HT\_ON});
    \item desligado (\textbf{HT\_OFF}).
\end{itemize*}

\vspace{0.5em}

Totalizando 144 combinações de configurações diferentes, com nossas entradas possuindo relacionamentos não-lineares entre si. Além disso, para configurações utilizando o \textit{Intel VTune Profiler} ele executa 5 vezes, e para configurações sem o \textit{Intel VTune Profiler} ele executa 10 vezes. Essa escolha foi devido ao \textit{overhead} do \textit{Intel VTune Profiler}, para dar tempo de realizarmos nossos experimentos dentro de um único \textit{batch} da ferramenta slurm.

As métricas que decidimos analisar estatísticamente em nosso trabalho utilizando o \textit{profiler} foram: \textit{Memory Bound}, \textit{CPI Rate}, e \textit{Average CPU Frequency}.
A métrica \textit{Memory Bound} é uma porcentagem dos ciclos da execução do programa onde o processador fica esperando operações na memória serem concluídas. A métrica \textit{CPI Rate} é a quantidade de ciclos por instrução média calculada durante toda a execução da aplicação. Por fim, a métrica \textit{Average CPU Frequency} é uma média simples da frequência de operação do processador durante a execução.

Por fim, versionamos nosso código em um repositório \textit{git} disponível em \url{https://github.com/thgdsg/perf-analysis}.

\section{Resultados}
\subsection{Métricas Brutas do \textit{Intel VTune Profiler}}
Vamos iniciar comentando sobre as métricas coletadas pelo \textit{Intel Vtune Profiler}. O gráfico abaixo é um agrupamento da frequência média de execução do programa, separando por grafo de entrada, quantidade de \textit{threads}, estado do \textit{Hyperthreading}, e política de vinculação de \textit{threads}.

\begin{figure}[h]
  \centering
  \includegraphics[width=\linewidth]{samples/gráficos/combined_average_cpu_frequency_all_graphs_scatter.pdf}
  \caption{Gráfico da Frequência Média da CPU por quantidade de \textit{threads} por configuração}
  \label{grafico_cpu_freq}
\end{figure}

No gráfico \ref{grafico_cpu_freq} podemos observar um comportamento similar a uma função exponencial decrescente, com grandes quedas na frequência média nos primeiros aumentos na quantidade de \textit{threads} de execução. Além disso, observa-se que executando com apenas 1 \textit{thread} em todos os casos a frequência média fica próxima a $3.6GHz$, o que é a frequência esperada de \textit{Turbo Boost} do processador. E a partir de 22 \textit{threads}, o processador executa em frequências abaixo de $3.0GHz$. Como ele possui 22 núcleos físicos, essa configuração utiliza todos os núcleos, e por isso atinge limites de gasto energético (Watts) e de temperatura (Celsius), tendo que limitar a frequência de operação dos núcleos para evitar desgastes no processador. Esse métrica nos confirma que o processador estava operando com o máximo de desempenho possível durante a execução de nossa aplicação, e valida nossos outros resultados obtidos.

\begin{table}[ht]
\centering
\small
\begin{tabular}{llrrrrrrrrr}
\toprule
\multirow{2}*{GRAPH\_NAME} & \multirow{2}*{CONFIG} & \multicolumn{9}{c}{THREADS} \\
\cmidrule(lr){3-11}
 & & 1 & 4 & 12 & 22 & 28 & 36 & 44 & 66 & 88 \\
\midrule
\multirow{4}{*}{BerkStan} & H\_OFF\_C & \cellcolor{ForestGreen}{9.56} & 14.38 & 13.60 & 14.00 & 15.08 & 14.70 & 16.12 & \cellcolor{BrickRed}{25.90} & 23.86 \\
 & H\_OFF\_S & \cellcolor{ForestGreen}{9.32} & 17.98 & 18.14 & 16.08 & 15.26 & 13.70 & 15.92 & \cellcolor{BrickRed}{25.06} & 23.16 \\
 & H\_ON\_C & 10.12 & \cellcolor{ForestGreen}{4.32} & 6.14 & 9.00 & 9.72 & 10.24 & 10.52 & 12.28 & \cellcolor{BrickRed}{13.30} \\
 & H\_ON\_S & \cellcolor{ForestGreen}{9.04} & 18.90 & \cellcolor{BrickRed}{19.08} & 16.70 & 14.96 & 14.82 & 14.94 & 10.72 & 13.14 \\
\cline{1-11}
\multirow{4}{*}{Friendster} & H\_OFF\_C & \cellcolor{ForestGreen}{29.20} & 33.08 & 41.74 & 56.46 & 66.58 & 70.16 & 73.02 & 73.30 & \cellcolor{BrickRed}{76.84} \\
 & H\_OFF\_S & \cellcolor{ForestGreen}{29.26} & 40.62 & 52.08 & 62.50 & 67.56 & 70.90 & 73.06 & 73.66 & \cellcolor{BrickRed}{76.20} \\
 & H\_ON\_C & 33.50 & 21.56 & \cellcolor{ForestGreen}{19.72} & 23.86 & 31.62 & 37.00 & 39.58 & 52.86 & \cellcolor{BrickRed}{68.36} \\
 & H\_ON\_S & \cellcolor{ForestGreen}{28.34} & 37.18 & 53.38 & 61.76 & 64.26 & 69.26 & \cellcolor{BrickRed}{71.50} & 61.56 & 68.00 \\
\cline{1-11}
\multirow{4}{*}{Google} & H\_OFF\_C & 30.14 & 28.22 & \cellcolor{ForestGreen}{25.26} & 26.56 & 32.86 & 30.52 & 27.36 & 51.34 & \cellcolor{BrickRed}{54.28} \\
 & H\_OFF\_S & 30.12 & 41.28 & 37.56 & 30.20 & 35.78 & 30.50 & \cellcolor{ForestGreen}{28.56} & 50.56 & \cellcolor{BrickRed}{53.44} \\
 & H\_ON\_C & 31.86 & \cellcolor{BrickRed}{32.06} & 29.26 & 24.34 & 25.66 & \cellcolor{ForestGreen}{22.06} & 23.30 & 27.16 & 24.06 \\
 & H\_ON\_S & 31.06 & \cellcolor{BrickRed}{41.20} & 36.14 & 31.02 & 34.06 & 29.24 & 30.60 & 27.24 & \cellcolor{ForestGreen}{26.76} \\
\cline{1-11}
\multirow{4}{*}{LiveJournal} & H\_OFF\_C & \cellcolor{ForestGreen}{17.92} & 23.26 & 28.86 & 35.18 & 34.40 & 24.70 & 37.14 & 53.06 & \cellcolor{BrickRed}{55.00} \\
 & H\_OFF\_S & \cellcolor{ForestGreen}{18.78} & 39.52 & 37.72 & 27.12 & 19.16 & 27.22 & 38.18 & 52.30 & \cellcolor{BrickRed}{63.32} \\
 & H\_ON\_C & 25.98 & 17.50 & 17.42 & \cellcolor{ForestGreen}{14.18} & 19.50 & \cellcolor{BrickRed}{29.40} & 26.22 & 25.10 & 28.94 \\
 & H\_ON\_S & 34.28 & 41.40 & 32.40 & \cellcolor{ForestGreen}{32.06} & 39.12 & 41.84 & \cellcolor{BrickRed}{45.62} & 35.38 & 34.00 \\
\cline{1-11}
\multirow{4}{*}{NotreDame} & H\_OFF\_C & \cellcolor{ForestGreen}{6.68} & 10.30 & 11.44 & 19.28 & 8.84 & 14.08 & 17.48 & 39.74 & \cellcolor{BrickRed}{46.60} \\
 & H\_OFF\_S & \cellcolor{ForestGreen}{6.58} & 21.50 & 17.32 & 16.22 & 13.88 & 14.58 & 15.98 & 36.82 & \cellcolor{BrickRed}{47.60} \\
 & H\_ON\_C & \cellcolor{ForestGreen}{5.48} & 8.06 & 9.00 & 5.78 & 6.72 & \cellcolor{BrickRed}{10.58} & 6.90 & 8.66 & 8.34 \\
 & H\_ON\_S & \cellcolor{ForestGreen}{6.30} & \cellcolor{BrickRed}{22.28} & 17.08 & 19.20 & 13.28 & 13.24 & 14.68 & 8.02 & 7.78 \\
\cline{1-11}
\multirow{4}{*}{Orkut} & H\_OFF\_C & 33.36 & 33.82 & \cellcolor{ForestGreen}{32.70} & 33.66 & 44.78 & 43.50 & 37.40 & \cellcolor{BrickRed}{60.12} & 60.00 \\
 & H\_OFF\_S & \cellcolor{ForestGreen}{33.00} & 50.20 & 44.56 & 43.42 & 43.38 & 40.50 & 38.38 & 59.42 & \cellcolor{BrickRed}{62.92} \\
 & H\_ON\_C & \cellcolor{BrickRed}{39.28} & 35.38 & 32.34 & 30.94 & 33.76 & \cellcolor{ForestGreen}{28.48} & 29.16 & 32.42 & 34.46 \\
 & H\_ON\_S & 32.14 & \cellcolor{BrickRed}{48.22} & 45.40 & 44.42 & 44.94 & 40.30 & 37.74 & 29.70 & \cellcolor{ForestGreen}{28.26} \\
\cline{1-11}
\multirow{4}{*}{Stanford} & H\_OFF\_C & 29.60 & 32.08 & 26.22 & 20.92 & 28.74 & 24.20 & \cellcolor{ForestGreen}{18.94} & 38.24 & \cellcolor{BrickRed}{41.08} \\
 & H\_OFF\_S & 29.96 & 40.52 & 35.34 & 26.64 & 26.84 & 23.54 & \cellcolor{ForestGreen}{19.88} & \cellcolor{BrickRed}{40.86} & 40.48 \\
 & H\_ON\_C & 29.18 & \cellcolor{BrickRed}{31.00} & 24.68 & 24.00 & 18.82 & \cellcolor{ForestGreen}{15.64} & 18.88 & 18.82 & 16.84 \\
 & H\_ON\_S & 29.68 & \cellcolor{BrickRed}{39.54} & 33.78 & 30.18 & 30.50 & 25.06 & 25.22 & \cellcolor{ForestGreen}{14.14} & 15.64 \\
\bottomrule
\end{tabular}
\caption{Tabela Combinada de Memory Bound (\%) (Valores Ótimos Destacados)}
\label{tab:memory-bound-combined-highlighted}
\end{table}


A tabela \ref{tab:memory-bound-combined-highlighted} mostra a métrica \textit{Memory Bound} pra todos os grafos e configurações combinados. Por exemplo, \textbf{H\_OFF\_C} se refere a configuração \textit{Hyperthreading} desligado com a política de vinculação de \textit{threads close}, podendo variar de 1 a 88 \textit{threads}. A tabela \ref{tab:cpi-rate-combined-highlighted} é organizada similarmente a tabela \ref{tab:memory-bound-combined-highlighted}.

Como é possível observar na tabela, nenhuma única configuração pra todos os grafos de entrada obteve a melhor ou pior taxa de ciclos ociosos esperando por operações na memória. Isso é particularmente interessante para nossa pesquisa pois uma das nossas hipóteses iniciais era que certas configurações poderiam aumentar bastante a quantidade de uso da memória, o que não foi o caso. Vale acrescentar que o coeficiente de variação médio entre todos os casos foi de aproximadamente 8\%, com pouca variação dessa métrica no geral entre \textit{runs} diferentes na mesma configuração.

\begin{table}[ht]
\centering
\small
\begin{tabular}{llrrrrrrrrr}
\toprule
\multirow{2}*{GRAPH\_NAME} & \multirow{2}*{CONFIG} & \multicolumn{9}{c}{THREADS} \\
\cmidrule(lr){3-11}
 & & 1 & 4 & 12 & 22 & 28 & 36 & 44 & 66 & 88 \\
\midrule
\multirow{4}{*}{BerkStan} & H\_OFF\_C & \cellcolor{ForestGreen}{0.539} & 0.563 & 0.678 & 0.838 & 0.955 & 0.996 & \cellcolor{BrickRed}{1.113} & 0.673 & 0.687 \\
 & H\_OFF\_S & \cellcolor{ForestGreen}{0.538} & 0.609 & 0.719 & 0.862 & 0.963 & 1.003 & \cellcolor{BrickRed}{1.113} & 0.674 & 0.689 \\
 & H\_ON\_C & \cellcolor{ForestGreen}{0.549} & 0.731 & 0.859 & 1.029 & 1.148 & 1.184 & 1.300 & 1.565 & \cellcolor{BrickRed}{1.783} \\
 & H\_ON\_S & \cellcolor{ForestGreen}{0.536} & 0.601 & 0.710 & 0.859 & 0.957 & 0.992 & 1.107 & 1.501 & \cellcolor{BrickRed}{1.791} \\
\cline{1-11}
\multirow{4}{*}{Friendster} & H\_OFF\_C & \cellcolor{ForestGreen}{0.663} & 0.697 & 0.861 & 1.314 & 2.020 & 2.590 & 3.104 & 2.975 & \cellcolor{BrickRed}{3.233} \\
 & H\_OFF\_S & \cellcolor{ForestGreen}{0.667} & 0.850 & 1.173 & 1.692 & 2.058 & 2.560 & 3.113 & 2.962 & \cellcolor{BrickRed}{3.226} \\
 & H\_ON\_C & \cellcolor{ForestGreen}{0.792} & 1.148 & 1.202 & 1.330 & 1.592 & 1.925 & 2.302 & 3.830 & \cellcolor{BrickRed}{5.187} \\
 & H\_ON\_S & \cellcolor{ForestGreen}{0.612} & 0.730 & 1.101 & 1.539 & 1.843 & 2.375 & 2.995 & 4.993 & \cellcolor{BrickRed}{5.845} \\
\cline{1-11}
\multirow{4}{*}{Google} & H\_OFF\_C & \cellcolor{ForestGreen}{0.936} & 0.960 & 0.995 & 1.100 & 1.362 & 1.464 & \cellcolor{BrickRed}{1.591} & 1.412 & 1.484 \\
 & H\_OFF\_S & \cellcolor{ForestGreen}{0.935} & 1.246 & 1.248 & 1.306 & 1.369 & 1.458 & \cellcolor{BrickRed}{1.626} & 1.363 & 1.413 \\
 & H\_ON\_C & \cellcolor{ForestGreen}{0.973} & 1.747 & 1.701 & 1.712 & 1.742 & 1.823 & 1.926 & 2.153 & \cellcolor{BrickRed}{2.271} \\
 & H\_ON\_S & \cellcolor{ForestGreen}{0.954} & 1.227 & 1.228 & 1.286 & 1.335 & 1.425 & 1.682 & 2.015 & \cellcolor{BrickRed}{2.288} \\
\cline{1-11}
\multirow{4}{*}{LiveJournal} & H\_OFF\_C & \cellcolor{ForestGreen}{0.589} & 0.692 & 0.926 & 1.262 & 1.340 & 1.479 & \cellcolor{BrickRed}{1.539} & 1.514 & 1.400 \\
 & H\_OFF\_S & \cellcolor{ForestGreen}{0.586} & 0.856 & 0.974 & 1.094 & 1.170 & 1.408 & \cellcolor{BrickRed}{1.554} & 1.505 & 1.399 \\
 & H\_ON\_C & \cellcolor{ForestGreen}{0.633} & 1.073 & 1.139 & 1.336 & 1.525 & 1.790 & 2.026 & 2.127 & \cellcolor{BrickRed}{2.411} \\
 & H\_ON\_S & \cellcolor{ForestGreen}{0.698} & 0.871 & 0.986 & 1.181 & 1.382 & 1.675 & 1.900 & 2.649 & \cellcolor{BrickRed}{3.154} \\
\cline{1-11}
\multirow{4}{*}{NotreDame} & H\_OFF\_C & \cellcolor{ForestGreen}{0.484} & 0.534 & 0.789 & 1.001 & 1.179 & 1.328 & \cellcolor{BrickRed}{1.422} & 0.752 & 0.793 \\
 & H\_OFF\_S & \cellcolor{ForestGreen}{0.485} & 0.632 & 0.855 & 1.053 & 1.176 & 1.326 & \cellcolor{BrickRed}{1.432} & 0.756 & 0.797 \\
 & H\_ON\_C & \cellcolor{ForestGreen}{0.487} & 0.752 & 1.064 & 1.272 & 1.414 & 1.561 & 1.705 & 1.992 & \cellcolor{BrickRed}{2.146} \\
 & H\_ON\_S & \cellcolor{ForestGreen}{0.483} & 0.628 & 0.856 & 1.053 & 1.176 & 1.320 & 1.429 & 1.897 & \cellcolor{BrickRed}{2.189} \\
\cline{1-11}
\multirow{4}{*}{Orkut} & H\_OFF\_C & \cellcolor{ForestGreen}{0.706} & 0.835 & 0.873 & 1.057 & 1.322 & 1.392 & \cellcolor{BrickRed}{1.465} & 1.371 & 1.416 \\
 & H\_OFF\_S & \cellcolor{ForestGreen}{0.712} & 1.136 & 1.172 & 1.170 & 1.250 & 1.327 & \cellcolor{BrickRed}{1.469} & 1.395 & 1.393 \\
 & H\_ON\_C & \cellcolor{ForestGreen}{0.830} & 1.301 & 1.364 & 1.432 & 1.496 & 1.430 & 1.611 & 2.051 & \cellcolor{BrickRed}{2.415} \\
 & H\_ON\_S & \cellcolor{ForestGreen}{0.688} & 1.073 & 1.134 & 1.163 & 1.254 & 1.343 & 1.472 & 1.842 & \cellcolor{BrickRed}{2.301} \\
\cline{1-11}
\multirow{4}{*}{Stanford} & H\_OFF\_C & \cellcolor{ForestGreen}{0.883} & 0.953 & 1.075 & 1.182 & 1.465 & 1.562 & \cellcolor{BrickRed}{1.665} & 1.217 & 1.262 \\
 & H\_OFF\_S & \cellcolor{ForestGreen}{0.886} & 1.178 & 1.286 & 1.341 & 1.451 & 1.568 & \cellcolor{BrickRed}{1.660} & 1.222 & 1.267 \\
 & H\_ON\_C & \cellcolor{ForestGreen}{0.889} & 1.586 & 1.627 & 1.663 & 1.765 & 1.890 & 2.002 & 2.291 & \cellcolor{BrickRed}{2.392} \\
 & H\_ON\_S & \cellcolor{ForestGreen}{0.883} & 1.174 & 1.273 & 1.333 & 1.451 & 1.563 & 1.657 & 2.154 & \cellcolor{BrickRed}{2.419} \\
\bottomrule
\end{tabular}
\caption{Tabela Combinada de CPI Rate (Valores Ótimos Destacados)}
\label{tab:cpi-rate-combined-highlighted}
\end{table}


Diferentemente da taxa de \textit{Memory Bound}, a taxa de ciclos por instrução na tabela \ref{tab:cpi-rate-combined-highlighted} possui um padrão bem claro, atingindo seu menor valor executando na configuração \textit{single-threaded}, e atingindo seu maior valor executando com 44 ou 88 \textit{threads}. Essa diferenciação entre maior valor em alguns casos está associada a configuração de \textit{Hyperthreading}, pois com ele desligado ele atinge o máximo de \textit{threads} de execução em 44, mas com o \textit{Hyperthreading} ligado o máximo de \textit{threads} é atingido em 88. 

Isso apenas não acontece no grafo Friendster, e nossa hipótese é que isso pode ser devido ao grande tamanho do grafo e altos volumes de transferência de memória nessas quantidades de \textit{threads}, algo que pode ser visto devido aos altos valores (>70\%) de \textit{Memory Bound} desse caso na tabela \ref{tab:memory-bound-combined-highlighted}.

\FloatBarrier
\subsection{Análise do \textit{Speedup}}
% --- Figura 1: com-Friendster ---
\begin{figure}[htbp]
    \centering
    \includegraphics[width=1.0\linewidth]{samples/gráficos/speedup_per_run_with_ci_com-Friendster.pdf}
    \caption{Análise de Speedup para o grafo com-Friendster.}
    \label{fig:friendster}
\end{figure}

% --- Figura 2: web-NotreDame ---
\begin{figure}[htbp]
    \centering
    \includegraphics[width=1.0\linewidth]{samples/gráficos/speedup_per_run_with_ci_web-NotreDame.pdf}
    \caption{Análise de Speedup para o grafo web-NotreDame.}
    \label{fig:notredame}
\end{figure}

% --- Figura 3: web-Stanford ---
\begin{figure}[htbp]
    \centering
    \includegraphics[width=1.0\linewidth]{samples/gráficos/speedup_per_run_with_ci_web-Stanford.pdf}
    \caption{Análise de Speedup para o grafo web-Stanford.}
    \label{fig:stanford}
\end{figure}

% --- Figura 4: com-LiveJournal ---
\begin{figure}[htbp]
    \centering
    \includegraphics[width=1.0\linewidth]{samples/gráficos/speedup_per_run_with_ci_com-LiveJournal.pdf}
    \caption{Análise de Speedup para o grafo com-LiveJournal.}
    \label{fig:livejournal}
\end{figure}

% --- Figura 5: web-Google ---
\begin{figure}[htbp]
    \centering
    \includegraphics[width=1.0\linewidth]{samples/gráficos/speedup_per_run_with_ci_web-Google.pdf}
    \caption{Análise de Speedup para o grafo web-Google.}
    \label{fig:google}
\end{figure}

% --- Figura 6: com-Orkut ---
\begin{figure}[htbp]
    \centering
    \includegraphics[width=1.0\linewidth]{samples/gráficos/speedup_per_run_with_ci_com-Orkut.pdf}
    \caption{Análise de Speedup para o grafo com-Orkut.}
    \label{fig:orkut}
\end{figure}

% --- Figura 7: web-BerkStan ---
\begin{figure}[htbp]
    \centering
    \includegraphics[width=1.0\linewidth]{samples/gráficos/speedup_per_run_with_ci_web-BerkStan.pdf}
    \caption{Análise de Speedup para o grafo web-BerkStan.}
    \label{fig:berkstan}
\end{figure}

Agora iremos comentar sobre o \textit{speedup} obtido executando sem o \textit{Intel Vtune Profiler}, coletado diretamente da aplicação.

Nas figuras \ref{fig:friendster} a \ref{fig:livejournal} é possível observar o \textit{speedup} obtido conforme o número de \textit{threads} foi aumentando, pra cada grafo de entrada. Além disso, nesse gráfico é possível ver o intervalo de confiança de 95\% de cada \textit{speedup} obtido. Na maioria dos gráficos, o intervalo de confiança é tão pequeno que é difícil de perceber, mas em grafos como web-Google e com-Orkut ele é perceptível.

Primeiramente, é possível observar que em algumas das figuras (\ref{fig:friendster}, \ref{fig:notredame}, \ref{fig:stanford}, e \ref{fig:livejournal}) o maior \textit{speedup} obtido foi com a configuração \textbf{HT\_OFF\_close}, e nos casos dos grafos com-Friendster, web-NotreDame, e web-Stanford essa configuração foi executada com 22 \textit{threads}. Acreditamos que o \textit{Hyperthreading} foi o maior responsável por isso, mas que a política de vinculação de \textit{threads close} também pode ter contribuído. 
Primeiro, executar com 22 \textit{threads} com o \textit{Hyperthreading} desligado está utilizando 100\% da capacidade de um processador físico, sem utilizar \textit{threads} lógicas, que poderiam dividir a memória \textit{cache} do núcleo. Além disso, a política de vinculação de \textit{threads close} nesses casos foi eficaz, conseguindo aproveitar a localidade espacial para obter ganhos no desempenho.

Também é possível observar nas figuras \ref{fig:notredame}, \ref{fig:stanford}, \ref{fig:livejournal}, \ref{fig:google}, e \ref{fig:berkstan} que a política \textit{close} resultou nos menores \textit{speedups} executando com entre 4 a 44 \textit{threads}. Como as \textit{threads} foram alocadas preferencialmente nos mesmos núcleos, isso gerou uma competição por acessos a memória \textit{cache} L2 e L3 (nosso processador não possui memória L1), diminuindo o desempenho atingido por essa configuração. Isto não aconteceu com o grafo web-Friendster e com-Orkut, e nossa hipótese é que foi devido a alta quantidade de arestas proporcionais aos vértices, com o grafo web-Friendster possuindo quase $28\times$ a quantidade de arestas, e o grafo com-Orkut possuindo pouco menos de $39\times$.

\clearpage

\subsection{Análise estatística}

Por fim, faremos a análise estatística dos nossos resultados, tentando correlacionar algumas variáveis de saída entre si e algumas das nossas variáveis de entrada com as de saída.

\begin{figure}[h]
  \centering
  \includegraphics[width=\linewidth]{samples/gráficos/correlacao_spearman.pdf}
  \caption{Gráfico da Correlação de Spearman entre as variáveis}
  \label{grafico_spearman}
\end{figure}

Primeiramente, na figura \ref{grafico_spearman} podemos observar a correlação de \textit{Spearman} entre as variáveis de saída. Consideramos o \textit{Speedup}, \textit{Memory Bound}, \textit{CPI Rate}, \textit{Average CPU Frequency}, e \textit{Parallel Efficiency}. A eficiência paralela é calculada dividindo o \textit{Speedup} pela quantidade de \textit{threads}.

Podemos ver que a taxa de ciclos por instrução possui uma forte correlação positiva com o \textit{speedup} e negativa com a eficiência paralela, condizente com os dados apresentados anteriormente na tabela \ref{tab:cpi-rate-combined-highlighted}, pois conforme a quantidade de \textit{threads} aumentava, a taxa de ciclos por instrução também aumentava, com o \textit{speedup} aumentando junto até certo ponto (normalmente 22 \textit{threads}). Similarmente, o \textit{memory bound} possuiu uma fraca correlação com a taxa de ciclos por instrução, relacionado com a fraca tendência da aplicação em aumentar o tempo perdido em transferência de memória conforme a quantidade de \textit{threads} aumentava.

\begin{figure}[h]
  \centering
  \includegraphics[width=\linewidth]{samples/gráficos/feature_importance.pdf}
  \caption{Gráfico da \textit{Feature Importance} das variáveis calculada por \textit{Random Forest}}
  \label{grafico_feature_importance}
\end{figure}

Dito isso, tentamos realizar um cálculo da importância das variáveis de saída, exceto a eficiência paralela, pra tentar ver o quanto elas estão relacionadas com o \textit{Speedup} adquirido e o quanto elas estão possívelmente limitando o desempenho do programa. Ou seja, relacionando o comportamento da máquina com o desempenho final. Pra isso, utilizamos \textit{Random Forest} com 100 estimadores e semente de execução fixa, e chegamos no resultado visível na figura \ref{grafico_feature_importance}.

Assim, é possível observar que a frequência da CPU tem uma fortíssima importância pro \textit{Speedup} da aplicação, e também que o tempo gasto esperando a memória impacta mais do que os ciclos por instrução, indicando que otimizações na memória podem ser cruciais para melhorar a escalabilidade na aplicação.

Por fim, aplicamos ANOVA pra relacionar nossas variáveis de entrada com as de saída, tanto as métricas, quanto o \textit{speedup} obtido. Calculamos pras variáveis \textit{threads} como controle, \textit{Hyperthreading}, \textit{Binding}, e a interação entre \textit{Hyperthreading} e \textit{Binding}. A coluna \textit{sum\_sq} mostra a magnitude do efeito, a coluna $F$ é uma razão que mostra o quanto as diferenças são devido a variação na entrada (maior = melhor). A coluna \textit{p-value} mostra a probabilidade desse resultado ser aleatório, e está marcado em verde caso seja menor que 0,05 (5\%).

\begin{table}[ht]
\centering
\begin{tabular}{lrrl}
\toprule
 & sum\_sq & F & p-value \\
\midrule
C(THREADS) & 21.194140 & 764.778120 & \cellcolor{ForestGreen}{0.000000} \\
C(HT\_OFF) & 0.009423 & 2.720265 & 0.100390 \\
C(BINDING) & 0.040890 & 11.804060 & \cellcolor{ForestGreen}{0.000696} \\
C(HT\_OFF):C(BINDING) & 0.041835 & 12.076720 & \cellcolor{ForestGreen}{0.000606} \\
Residual & 0.831384 & NaN & - \\
\bottomrule
\end{tabular}
\caption{ANOVA para AVERAGE\_CPU\_FREQUENCY (GHz)}
\label{tab:anova_average_cpu_frequency}
\end{table}

Na tabela \ref{tab:anova_average_cpu_frequency} podemos ver que a política de vinculação de \textit{threads} afeta a frequência média. Uma possível razão disso é que manter a política em \textit{close} pode aumentar a temperatura do \textit{chip}, pois núcleos mais próximos fisicamente serão utilizados, atingindo temperaturas máximas mais rápido. E além disso, há interação entre \textit{hyperthreading} e a política escolhida, ou seja, provavelmente ligar o \textit{hyperthreading} e manter a política em \textit{close} seria o pior resultado térmico.

\begin{table}[ht]
\centering
\begin{tabular}{lrrl}
\toprule
 & sum\_sq & F & p-value \\
\midrule
C(THREADS) & 48.361651 & 18.586491 & \cellcolor{ForestGreen}{0.000000} \\
C(HT\_OFF) & 7.106524 & 21.849600 & \cellcolor{ForestGreen}{0.000005} \\
C(BINDING) & 0.033739 & 0.103734 & 0.747674 \\
C(HT\_OFF):C(BINDING) & 0.294290 & 0.904820 & 0.342450 \\
Residual & 78.059354 & NaN & - \\
\bottomrule
\end{tabular}
\caption{ANOVA para CPI\_RATE}
\label{tab:anova_cpi_rate}
\end{table}

Na tabela \ref{tab:anova_cpi_rate}, é possível ver pelo valor $F$ de 21 que o \textit{hyperthreading} possui um impacto tangível nos ciclos por instrução, provavelmente devido ao comportamento dos ciclos por instrução ser fortemente dependente da quantidade de \textit{threads} e da disputa por recursos dentro do núcleo, caso o \textit{Hyperthreading} esteja ativado. Além disso, não há interação visível entre o \textit{hyperthreading} e a política de vinculação pra essa métrica.

\begin{table}[ht]
\centering
\begin{tabular}{lrrl}
\toprule
 & sum\_sq & F & p-value \\
\midrule
C(THREADS) & 5300.901527 & 2.754524 & \cellcolor{ForestGreen}{0.006290} \\
C(HT\_OFF) & 3510.528229 & 14.593494 & \cellcolor{ForestGreen}{0.000170} \\
C(BINDING) & 2063.090006 & 8.576399 & \cellcolor{ForestGreen}{0.003733} \\
C(HT\_OFF):C(BINDING) & 672.672057 & 2.796342 & 0.095783 \\
Residual & 57733.038537 & NaN & - \\
\bottomrule
\end{tabular}
\caption{ANOVA para MEMORY\_BOUND}
\label{tab:anova_memory_bound}
\end{table}

Na tabela \ref{tab:anova_memory_bound} vemos que ambos \textit{hyperthreading} e a política de vinculação impactam a métrica de \textit{Memory Bound}, sendo isso especialmente evidente pelo valor $F$ do \textit{hyperthreading}. Isso é condizente com nossas figuras \ref{fig:notredame}, \ref{fig:stanford}, \ref{fig:livejournal}, e \ref{fig:google}, onde o estado do \textit{Hyperthreading} causou uma diminuição grande no \textit{speedup} da aplicação nesses casos.

\centering
\begin{tabular}{lrrl}
\toprule
 & sum\_sq & F & p-value \\
\midrule
C(THREADS) & 191.385561 & 17.909454 & \cellcolor{ForestGreen}{0.000000} \\
C(HT\_OFF) & 0.508229 & 0.380472 & 0.537936 \\
C(BINDING) & 1.135873 & 0.850341 & 0.357382 \\
C(HT\_OFF):C(BINDING) & 0.734943 & 0.550195 & 0.458963 \\
Residual & 320.588597 & NaN & - \\
\bottomrule
\end{tabular}
\label{tab:anova_speedup}
Por fim, na tabela \ref{tab:anova_speedup} podemos observar que tanto o \textit{hyperthreading} quanto a política de vinculação não possuem impacto significativo no \textit{speedup} da aplicação num contexto global. E além disso, elas não possuem interação estatísticamente relevante entre si nesse contexto.

Embora essa análise de variância global não tenha demonstrado nenhuma significância estatística do estado do \textit{hyperthreading} e da política de vinculação de \textit{threads} no \textit{speedup}, os resultados anteriores demonstram que há uma grande mudança no comportamento da máquina, e os resultados de cada grafo de entrada dependem da sensibilidade da entrada ao comportamento. Uma análise crítica com mais métricas seria necessária para gerar um modelo com maior certeza sobre o que afeta o desempenho da aplicação.

\section{Conclusão}
\subsection{Conclusões}
Este trabalho apresentou uma análise detalhada do desempenho do algoritmo PageRank em arquiteturas modernas, explorando o impacto de configurações de \textit{multithreading}, afinidade de processador e microarquitetura no tempo de execução. Através de uma metodologia experimental combinando métricas de tempo real com instrumentação via \textit{Intel VTune Profiler} e validação estatística via ANOVA, conseguimos identificar alguns gargalos que limitam a escalabilidade de algoritmos baseados em grafos.

Nossa análise demonstrou que para algoritmos limitados por memória como o PageRank a estratégia padrão de usar o máximo de \textit{threads} lógicas disponíveis nem sempre resulta no melhor desempenho. Observamos que configurações utilizando apenas núcleos físicos com \textit{Hyper-Threading} desligado (\textbf{HT\_OFF}) frequentemente superaram o uso de todas as \textit{threads} lógicas.

E com isso, nós chegamos a três conclusões principais:
\begin{itemize}
    \item Observamos um aumento na métrica do \textit{CPI Rate} e \textit{Memory Bound} ao ligarmos o \textit{hyperthreading}. Isso indica que o \textit{hyperthreading} aumentou a disputa por recursos de memória e \textit{caches} L2/L3, sem oferecer ganho de vazão suficiente para compensar a latência adicional. Na maioria dos grafos, executar com o \textit{hyperthreading} desativado foi a melhor configuração considerando o \textit{speedup}.
    
    \item A política de vinculação \textit{close} mostrou-se superior em diversos cenários, especialmente quando o número de \textit{threads} se limitava a um único soquete físico (menor que 44). Isso sugere que a comunicação entre processadores diferentes na mesma placa-mãe pode causar uma latência significativa na execução. No entanto, em grafos com estruturas de alta densidade (como o \textit{com-Orkut}), a política \textit{spread} pode ser necessária para evitar limitações térmicas do processador.

    \item E por fim, a análise estatística juntamente com os gráficos de \textit{speedup} nos demonstraram que não existe uma única configuração ideal independente da entrada. A eficiência da configuração nesse algoritmo depende altamente da sensibilidade da entrada ao estado atual do sistema.
\end{itemize}

\subsection{Aprendizados e Dificuldades}
Nesse trabalho, desenvolvemos as seguintes habilidades:
\begin{itemize}
\item Aprendemos a fazer análises estatísticas criteriosas.
\item Aprendemos a organizar nossos dados, e a inseri-los em uma plataforma reprodutível
\item Aprendemos a fazer pesquisa de maneira séria, considerando variações estatísticas e construindo modelos que minimizem o impacto dessas variações.
\end{itemize}

Além disso, essas foram nossas principais dificuldades:
\begin{itemize}
\item Organizar os dados gerados pelo \textit{Intel Vtune Profiler}. Ele gera muitos dados, e a maioria não era de interesse do nosso grupo.
\item A nossa máquina nem sempre estava disponível para ser alocada.
\item Desligar o \textit{hyperthreading} na máquina foi difícil, pois é necessário fazer manualmente, não havendo nenhum programa ou comando pronto que possa fazer isso.
\item Não estávamos acostumados a fazer pesquisa dessa maneira, utilizar \textit{notebooks} é algo novo pro grupo.
\end{itemize}

%%
%% The next two lines define the bibliography style to be used, and
%% the bibliography file.
\bibliographystyle{ACM-Reference-Format}
\bibliography{samples/refs}

\end{document}
\endinput
%%
%% End of file `sample-acmsmall.tex'.
