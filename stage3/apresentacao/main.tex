%% main.tex adaptado para o template Inf da UFRGS
%% Conteúdo original de João Vitor Spolavore e Thiago Gonçalves
%
% This work may be distributed and/or modified under the
% conditions of the LaTeX Project Public License, either version 1.3
% of this license or (at your option) any later version.
% The latest version of this license is in
%   http://www.latex-project.org/lppl.txt
% and version 1.3 or later is part of all distributions of LaTeX
% version 2005/12/01 or later.
\PassOptionsToPackage{table,dvipsnames}{xcolor}
\documentclass{beamer}
\usepackage[T1]{fontenc}
\usepackage[brazil]{babel}
\usepackage[utf8]{inputenc}
\usepackage{hyperref}
\usepackage{booktabs} % For better tables
\usepackage{graphicx}
\usepackage{booktabs} 
\usepackage{multirow}

% Choose the Inf theme
\usetheme{Inf}

% Define the title with \title[short title]{long title}
\title[Avaliação PageRank]{Avaliação de desempenho do algoritmo PageRank sequencial e paralelo}

\date{\today}

% Author information
\author{João Vitor Spolavore \and Thiago Gonçalves}
\institute{Instituto de Informática --- UFRGS}

\begin{document}

% Command to create title page
\InfTitlePage


\section{Introdução}
\begin{frame}{Introdução}
    O presente relatório tem como objetivo analisar o comportamento do sistema ao aplicar o algoritmo PageRank sob diferentes níveis de paralelismo, avaliando como o aumento do número de threads impacta seu desempenho.\\
    \vspace{0.5em}
\end{frame}

\section{Ambiente de Teste}
\begin{frame}{Máquina Utilizada}
    Todos os experimentos realizados foram executados no Parque de Computação de Alto Desempenho (PCAD) da Universidade Federal do Rio Grande do Sul (UFRGS).
    Utilizamos o nó computacional denominado \textit{blaise}.

    \subsection*{Características da Máquina}
    \begin{itemize}
        \item \textbf{Nome:} blaise
        \item \textbf{CPU:} 2 x Intel(R) Xeon(R) E5-2699 v4, 2.20 GHz, 88 threads, 44 cores
        \item \textbf{RAM:} 256GB 
        \item \textbf{Disco:} 1.8 TB SSD, 1.8 TB HDD
        \item \textbf{Placa mãe:} Supermicro X10DGQ
    \end{itemize}
    \url{https://gppd-hpc.inf.ufrgs.br}\\
    \vspace{0.6em}
    Esta máquina possui DOIS processadores Intel, fundamental para a escolha devido às possibilidades de experimentos com distribuição de threads.
\end{frame}

\section{Método de análise e coleta de dados}
\begin{frame}{Método de análise}
    O presente projeto apresenta uma perspectiva de medição para o método de análise.\\
    \vspace{0.5em}
    Para ajudar nessa abordagem, utilizamos o Intel Vtune Profiler para coletar métricas de hardware da aplicação. Decidimos utilizar essa ferramenta pois:
    \vspace{0.5em}
    \begin{itemize}
        \item Os processadores presentes na máquina escolhida eram Intel.
        \item É uma ferramenta amplamente utilizada pela própria Intel para obter métricas de seus processadores.
    \end{itemize}

\end{frame}

\begin{frame}{Definição dos experimentos}
    Para obter o tempo de execução da programa (\textit{Elapsed time}) não utilizamos o profiler da Intel. Uma vez que a sua utilização adiciona um \textit{overhead} significativo nessa métrica.\\
    \vspace{0.5em} 
    Executamos cada configuração 5 vezes utilizando o Vtune e 10 vezes sem a utilização do profiler. 
\end{frame}


\begin{frame}{Parâmetros Considerados}
    Consideramos as seguintes opções para análise:
    \begin{itemize}
        \item \textbf{Número de Threads}: {1, 4, 12, 22, 28, 36, 44, 66, 88}.
  \item \textbf{Estado do Hyperthreading (ON/OFF)}: Com o objetivo de avaliar o impacto no desempenho, forçamos 1 thread por core com a variável \texttt{OMP\_PLACES=cores/threads} e reduzimos visibilidade de threads lógicas com \texttt{GOMP\_CPU\_AFFINITY} e o comando \texttt{taskset -c [NÚMEROS DOS CORES]}.
        \item \textbf{Política de \textit{Binding} de Threads}:Distribuição de threads entre os núcleos dos processadores via \texttt{OMP\_PROC\_BIND} com essas opções:
        \begin{itemize}
            \item \texttt{close}: Vincula threads a lugares próximos à thread pai, útil para localidade de cache.
            \item \texttt{spread}: Distribui threads em partições mais distantes, útil para reduzir contenção em recursos compartilhados.
        \end{itemize}
    \end{itemize}
\end{frame}


\section{Algoritmo e Dados}

\begin{frame}
    \frametitle{Algoritmo e Fonte de Dados}

    \begin{itemize}
        \item \textbf{Algoritmo:} Utilizamos uma implementação consolidada do PageRank do repositório GAPBS (\url{https://github.com/sbeamer/gapbs}).
        \item \textbf{Fonte dos Dados:} Os grafos foram obtidos da \textit{Stanford Large Network Dataset Collection} (SNAP), um repositório consagrado academicamente (\url{https://snap.stanford.edu/data/index.html}).
        \item \textbf{Requisito:} Para executar o algoritmo PageRank, a aplicação recebe como entrada grafos \textbf{direcionados}.
    \end{itemize}
\end{frame}

\begin{frame}
    \frametitle{Grafos Utilizados}

    A tabela abaixo detalha as redes de grafos escolhidas para os experimentos:

    \centering
    \begin{tabular}{lrr}
        \toprule
        \textbf{Grafo} & \textbf{Vértices} & \textbf{Arestas} \\
        \midrule
        Friendster     & 65,608,366        & 1,806,067,135    \\
        LiveJournal    & 3,997,962         & 34,681,189       \\
        Orkut          & 3,072,441         & 117,185,083      \\
        BerkStan       & 685,230           & 7,600,595        \\
        Google         & 875,713           & 5,105,039        \\
        NotreDame      & 325,729           & 1,497,134        \\
        Stanford       & 281,903           & 2,312,497        \\
        \bottomrule
    \end{tabular}

    \vspace{1em} % Adiciona um espaço vertical
    Todos representam conexões entre páginas web e/ou redes sociais e são dados do mundo real.

\end{frame}

\section{Resultados}

\begin{frame}{Métricas brutas do \textit{Intel VTune Profiler}}
        Primeiro, mostraremos as métricas brutas do \textit{Intel Vtune Profiler}.
        Vamos iniciar vendo o gráfico de Frequência Média da CPU durante a execução do algoritmo.

        Nesse gráfico, os pontos representam todos os grafos de entrada, com as 4 seguintes configurações:
        \begin{enumerate}
            \item Hyperthreading Desligado, Política Spread - \textbf{HT\_OFF\_spread}
            \item Hyperthreading Ligado, Política Spread - \textbf{HT\_ON\_spread}
            \item Hyperthreading Desligado, Política Close - \textbf{HT\_OFF\_close}
            \item Hyperthreading Ligado, Política Close - \textbf{HT\_ON\_close}
        \end{enumerate}
\end{frame}

\begin{frame}
    \frametitle{Frequência Média da CPU}

    \begin{figure}
        \centering
        \makebox[\textwidth][c]{%

        \includegraphics[width=0.95\paperwidth]{graficos/combined_average_cpu_frequency_all_graphs_scatter.pdf}
        }
    \end{figure}
    % insira comentario aqui
    \centering
\end{frame}

\begin{frame}{Explicação das Tabelas Combinadas}
        Como o tempo de nossa apresentação é limitado, apenas apresentaremos a análise para uma entrada da aplicação.

        Agora mostraremos as tabelas de \textit{Memory Bound} e ciclos por instrução.
        
        \textit{Memory Bound} é a porcentagem de tempo perdido esperando operações na memória.

        Ciclos por instrução (CPI) é uma média total de ciclos pra cada instrução do programa.
        
\end{frame}

\begin{frame}{Tabela de \textit{Memory Bound} (\%)}
    \resizebox{\textwidth}{!}{%
        \begin{table}[ht]
\centering
\small
\begin{tabular}{llrrrrrrrrr}
\toprule
\multirow{2}*{GRAPH\_NAME} & \multirow{2}*{CONFIG} & \multicolumn{9}{c}{THREADS} \\
\cmidrule(lr){3-11}
 & & 1 & 4 & 12 & 22 & 28 & 36 & 44 & 66 & 88 \\
\midrule
\multirow{4}{*}{BerkStan} & H\_OFF\_C & \cellcolor{ForestGreen}{9.56} & 14.38 & 13.60 & 14.00 & 15.08 & 14.70 & 16.12 & \cellcolor{BrickRed}{25.90} & 23.86 \\
 & H\_OFF\_S & \cellcolor{ForestGreen}{9.32} & 17.98 & 18.14 & 16.08 & 15.26 & 13.70 & 15.92 & \cellcolor{BrickRed}{25.06} & 23.16 \\
 & H\_ON\_C & 10.12 & \cellcolor{ForestGreen}{4.32} & 6.14 & 9.00 & 9.72 & 10.24 & 10.52 & 12.28 & \cellcolor{BrickRed}{13.30} \\
 & H\_ON\_S & \cellcolor{ForestGreen}{9.04} & 18.90 & \cellcolor{BrickRed}{19.08} & 16.70 & 14.96 & 14.82 & 14.94 & 10.72 & 13.14 \\
\cline{1-11}
\multirow{4}{*}{Friendster} & H\_OFF\_C & \cellcolor{ForestGreen}{29.20} & 33.08 & 41.74 & 56.46 & 66.58 & 70.16 & 73.02 & 73.30 & \cellcolor{BrickRed}{76.84} \\
 & H\_OFF\_S & \cellcolor{ForestGreen}{29.26} & 40.62 & 52.08 & 62.50 & 67.56 & 70.90 & 73.06 & 73.66 & \cellcolor{BrickRed}{76.20} \\
 & H\_ON\_C & 33.50 & 21.56 & \cellcolor{ForestGreen}{19.72} & 23.86 & 31.62 & 37.00 & 39.58 & 52.86 & \cellcolor{BrickRed}{68.36} \\
 & H\_ON\_S & \cellcolor{ForestGreen}{28.34} & 37.18 & 53.38 & 61.76 & 64.26 & 69.26 & \cellcolor{BrickRed}{71.50} & 61.56 & 68.00 \\
\cline{1-11}
\multirow{4}{*}{Google} & H\_OFF\_C & 30.14 & 28.22 & \cellcolor{ForestGreen}{25.26} & 26.56 & 32.86 & 30.52 & 27.36 & 51.34 & \cellcolor{BrickRed}{54.28} \\
 & H\_OFF\_S & 30.12 & 41.28 & 37.56 & 30.20 & 35.78 & 30.50 & \cellcolor{ForestGreen}{28.56} & 50.56 & \cellcolor{BrickRed}{53.44} \\
 & H\_ON\_C & 31.86 & \cellcolor{BrickRed}{32.06} & 29.26 & 24.34 & 25.66 & \cellcolor{ForestGreen}{22.06} & 23.30 & 27.16 & 24.06 \\
 & H\_ON\_S & 31.06 & \cellcolor{BrickRed}{41.20} & 36.14 & 31.02 & 34.06 & 29.24 & 30.60 & 27.24 & \cellcolor{ForestGreen}{26.76} \\
\cline{1-11}
\multirow{4}{*}{LiveJournal} & H\_OFF\_C & \cellcolor{ForestGreen}{17.92} & 23.26 & 28.86 & 35.18 & 34.40 & 24.70 & 37.14 & 53.06 & \cellcolor{BrickRed}{55.00} \\
 & H\_OFF\_S & \cellcolor{ForestGreen}{18.78} & 39.52 & 37.72 & 27.12 & 19.16 & 27.22 & 38.18 & 52.30 & \cellcolor{BrickRed}{63.32} \\
 & H\_ON\_C & 25.98 & 17.50 & 17.42 & \cellcolor{ForestGreen}{14.18} & 19.50 & \cellcolor{BrickRed}{29.40} & 26.22 & 25.10 & 28.94 \\
 & H\_ON\_S & 34.28 & 41.40 & 32.40 & \cellcolor{ForestGreen}{32.06} & 39.12 & 41.84 & \cellcolor{BrickRed}{45.62} & 35.38 & 34.00 \\
\cline{1-11}
\multirow{4}{*}{NotreDame} & H\_OFF\_C & \cellcolor{ForestGreen}{6.68} & 10.30 & 11.44 & 19.28 & 8.84 & 14.08 & 17.48 & 39.74 & \cellcolor{BrickRed}{46.60} \\
 & H\_OFF\_S & \cellcolor{ForestGreen}{6.58} & 21.50 & 17.32 & 16.22 & 13.88 & 14.58 & 15.98 & 36.82 & \cellcolor{BrickRed}{47.60} \\
 & H\_ON\_C & \cellcolor{ForestGreen}{5.48} & 8.06 & 9.00 & 5.78 & 6.72 & \cellcolor{BrickRed}{10.58} & 6.90 & 8.66 & 8.34 \\
 & H\_ON\_S & \cellcolor{ForestGreen}{6.30} & \cellcolor{BrickRed}{22.28} & 17.08 & 19.20 & 13.28 & 13.24 & 14.68 & 8.02 & 7.78 \\
\cline{1-11}
\multirow{4}{*}{Orkut} & H\_OFF\_C & 33.36 & 33.82 & \cellcolor{ForestGreen}{32.70} & 33.66 & 44.78 & 43.50 & 37.40 & \cellcolor{BrickRed}{60.12} & 60.00 \\
 & H\_OFF\_S & \cellcolor{ForestGreen}{33.00} & 50.20 & 44.56 & 43.42 & 43.38 & 40.50 & 38.38 & 59.42 & \cellcolor{BrickRed}{62.92} \\
 & H\_ON\_C & \cellcolor{BrickRed}{39.28} & 35.38 & 32.34 & 30.94 & 33.76 & \cellcolor{ForestGreen}{28.48} & 29.16 & 32.42 & 34.46 \\
 & H\_ON\_S & 32.14 & \cellcolor{BrickRed}{48.22} & 45.40 & 44.42 & 44.94 & 40.30 & 37.74 & 29.70 & \cellcolor{ForestGreen}{28.26} \\
\cline{1-11}
\multirow{4}{*}{Stanford} & H\_OFF\_C & 29.60 & 32.08 & 26.22 & 20.92 & 28.74 & 24.20 & \cellcolor{ForestGreen}{18.94} & 38.24 & \cellcolor{BrickRed}{41.08} \\
 & H\_OFF\_S & 29.96 & 40.52 & 35.34 & 26.64 & 26.84 & 23.54 & \cellcolor{ForestGreen}{19.88} & \cellcolor{BrickRed}{40.86} & 40.48 \\
 & H\_ON\_C & 29.18 & \cellcolor{BrickRed}{31.00} & 24.68 & 24.00 & 18.82 & \cellcolor{ForestGreen}{15.64} & 18.88 & 18.82 & 16.84 \\
 & H\_ON\_S & 29.68 & \cellcolor{BrickRed}{39.54} & 33.78 & 30.18 & 30.50 & 25.06 & 25.22 & \cellcolor{ForestGreen}{14.14} & 15.64 \\
\bottomrule
\end{tabular}
\caption{Tabela Combinada de Memory Bound (\%) (Valores Ótimos Destacados)}
\label{tab:memory-bound-combined-highlighted}
\end{table}

    }
    \begin{itemize}
    \item Nenhuma única entrada obteve a maior ou menor taxa de ciclos ociosos esperando por operações na memória, sendo um padrão que se repete ao longo de todas as entradas.

    \item Além disso, por mais que seja possível observar que nesse caso os melhores \textit{Memory Bound} estavam com menos \textit{threads} e os piores com mais, esse comportamento não se repete com todas as entradas.
    \end{itemize}
\end{frame}

\begin{frame}{Tabela de Ciclos por Instrução}
    \resizebox{\textwidth}{!}{%
        \begin{table}[ht]
\centering
\small
\begin{tabular}{llrrrrrrrrr}
\toprule
\multirow{2}*{GRAPH\_NAME} & \multirow{2}*{CONFIG} & \multicolumn{9}{c}{THREADS} \\
\cmidrule(lr){3-11}
 & & 1 & 4 & 12 & 22 & 28 & 36 & 44 & 66 & 88 \\
\midrule
\multirow{4}{*}{BerkStan} & H\_OFF\_C & \cellcolor{ForestGreen}{0.539} & 0.563 & 0.678 & 0.838 & 0.955 & 0.996 & \cellcolor{BrickRed}{1.113} & 0.673 & 0.687 \\
 & H\_OFF\_S & \cellcolor{ForestGreen}{0.538} & 0.609 & 0.719 & 0.862 & 0.963 & 1.003 & \cellcolor{BrickRed}{1.113} & 0.674 & 0.689 \\
 & H\_ON\_C & \cellcolor{ForestGreen}{0.549} & 0.731 & 0.859 & 1.029 & 1.148 & 1.184 & 1.300 & 1.565 & \cellcolor{BrickRed}{1.783} \\
 & H\_ON\_S & \cellcolor{ForestGreen}{0.536} & 0.601 & 0.710 & 0.859 & 0.957 & 0.992 & 1.107 & 1.501 & \cellcolor{BrickRed}{1.791} \\
\cline{1-11}
\multirow{4}{*}{Friendster} & H\_OFF\_C & \cellcolor{ForestGreen}{0.663} & 0.697 & 0.861 & 1.314 & 2.020 & 2.590 & 3.104 & 2.975 & \cellcolor{BrickRed}{3.233} \\
 & H\_OFF\_S & \cellcolor{ForestGreen}{0.667} & 0.850 & 1.173 & 1.692 & 2.058 & 2.560 & 3.113 & 2.962 & \cellcolor{BrickRed}{3.226} \\
 & H\_ON\_C & \cellcolor{ForestGreen}{0.792} & 1.148 & 1.202 & 1.330 & 1.592 & 1.925 & 2.302 & 3.830 & \cellcolor{BrickRed}{5.187} \\
 & H\_ON\_S & \cellcolor{ForestGreen}{0.612} & 0.730 & 1.101 & 1.539 & 1.843 & 2.375 & 2.995 & 4.993 & \cellcolor{BrickRed}{5.845} \\
\cline{1-11}
\multirow{4}{*}{Google} & H\_OFF\_C & \cellcolor{ForestGreen}{0.936} & 0.960 & 0.995 & 1.100 & 1.362 & 1.464 & \cellcolor{BrickRed}{1.591} & 1.412 & 1.484 \\
 & H\_OFF\_S & \cellcolor{ForestGreen}{0.935} & 1.246 & 1.248 & 1.306 & 1.369 & 1.458 & \cellcolor{BrickRed}{1.626} & 1.363 & 1.413 \\
 & H\_ON\_C & \cellcolor{ForestGreen}{0.973} & 1.747 & 1.701 & 1.712 & 1.742 & 1.823 & 1.926 & 2.153 & \cellcolor{BrickRed}{2.271} \\
 & H\_ON\_S & \cellcolor{ForestGreen}{0.954} & 1.227 & 1.228 & 1.286 & 1.335 & 1.425 & 1.682 & 2.015 & \cellcolor{BrickRed}{2.288} \\
\cline{1-11}
\multirow{4}{*}{LiveJournal} & H\_OFF\_C & \cellcolor{ForestGreen}{0.589} & 0.692 & 0.926 & 1.262 & 1.340 & 1.479 & \cellcolor{BrickRed}{1.539} & 1.514 & 1.400 \\
 & H\_OFF\_S & \cellcolor{ForestGreen}{0.586} & 0.856 & 0.974 & 1.094 & 1.170 & 1.408 & \cellcolor{BrickRed}{1.554} & 1.505 & 1.399 \\
 & H\_ON\_C & \cellcolor{ForestGreen}{0.633} & 1.073 & 1.139 & 1.336 & 1.525 & 1.790 & 2.026 & 2.127 & \cellcolor{BrickRed}{2.411} \\
 & H\_ON\_S & \cellcolor{ForestGreen}{0.698} & 0.871 & 0.986 & 1.181 & 1.382 & 1.675 & 1.900 & 2.649 & \cellcolor{BrickRed}{3.154} \\
\cline{1-11}
\multirow{4}{*}{NotreDame} & H\_OFF\_C & \cellcolor{ForestGreen}{0.484} & 0.534 & 0.789 & 1.001 & 1.179 & 1.328 & \cellcolor{BrickRed}{1.422} & 0.752 & 0.793 \\
 & H\_OFF\_S & \cellcolor{ForestGreen}{0.485} & 0.632 & 0.855 & 1.053 & 1.176 & 1.326 & \cellcolor{BrickRed}{1.432} & 0.756 & 0.797 \\
 & H\_ON\_C & \cellcolor{ForestGreen}{0.487} & 0.752 & 1.064 & 1.272 & 1.414 & 1.561 & 1.705 & 1.992 & \cellcolor{BrickRed}{2.146} \\
 & H\_ON\_S & \cellcolor{ForestGreen}{0.483} & 0.628 & 0.856 & 1.053 & 1.176 & 1.320 & 1.429 & 1.897 & \cellcolor{BrickRed}{2.189} \\
\cline{1-11}
\multirow{4}{*}{Orkut} & H\_OFF\_C & \cellcolor{ForestGreen}{0.706} & 0.835 & 0.873 & 1.057 & 1.322 & 1.392 & \cellcolor{BrickRed}{1.465} & 1.371 & 1.416 \\
 & H\_OFF\_S & \cellcolor{ForestGreen}{0.712} & 1.136 & 1.172 & 1.170 & 1.250 & 1.327 & \cellcolor{BrickRed}{1.469} & 1.395 & 1.393 \\
 & H\_ON\_C & \cellcolor{ForestGreen}{0.830} & 1.301 & 1.364 & 1.432 & 1.496 & 1.430 & 1.611 & 2.051 & \cellcolor{BrickRed}{2.415} \\
 & H\_ON\_S & \cellcolor{ForestGreen}{0.688} & 1.073 & 1.134 & 1.163 & 1.254 & 1.343 & 1.472 & 1.842 & \cellcolor{BrickRed}{2.301} \\
\cline{1-11}
\multirow{4}{*}{Stanford} & H\_OFF\_C & \cellcolor{ForestGreen}{0.883} & 0.953 & 1.075 & 1.182 & 1.465 & 1.562 & \cellcolor{BrickRed}{1.665} & 1.217 & 1.262 \\
 & H\_OFF\_S & \cellcolor{ForestGreen}{0.886} & 1.178 & 1.286 & 1.341 & 1.451 & 1.568 & \cellcolor{BrickRed}{1.660} & 1.222 & 1.267 \\
 & H\_ON\_C & \cellcolor{ForestGreen}{0.889} & 1.586 & 1.627 & 1.663 & 1.765 & 1.890 & 2.002 & 2.291 & \cellcolor{BrickRed}{2.392} \\
 & H\_ON\_S & \cellcolor{ForestGreen}{0.883} & 1.174 & 1.273 & 1.333 & 1.451 & 1.563 & 1.657 & 2.154 & \cellcolor{BrickRed}{2.419} \\
\bottomrule
\end{tabular}
\caption{Tabela Combinada de CPI Rate (Valores Ótimos Destacados)}
\label{tab:cpi-rate-combined-highlighted}
\end{table}

    }

    \begin{itemize}
    \item O comportamento de ciclos por instrução se repetiu ao longo de todas as entradas.

    \item O grafo Friendster foi uma exceção, com os outros grafos possuindo o maior CPI com o \textit{Hyperthreading} desligado ao executar com 44 \textit{threads}.

    \item Achamos que pode ser por causa do grande tamanho do grafo e altos volumes de transferência de dados.
    \end{itemize}
\end{frame}

\begin{frame}{Análise do \textit{Speedup}}
    O gráfico a seguir mostra o \textit{speedup} para o grafo com-Friendster, o maior grafo em que realizamos nossos experimentos, com intervalo de confiança de 95\% entre cada ponto.
    
    Nesse gráfico, os pontos representam todos os grafos de entrada, com as 4 seguintes configurações:
    \begin{enumerate}
        \item \textbf{HT\_OFF\_spread}
        \item \textbf{HT\_ON\_spread}
        \item \textbf{HT\_OFF\_close}
        \item \textbf{HT\_ON\_close}
    \end{enumerate}
\end{frame}

\begin{frame}{Speedup - Friendster}

    \begin{figure}
        \centering
        \makebox[\textwidth][c]{%
        \includegraphics[width=0.95\paperwidth]{graficos/speedup_per_run_with_ci_com-Friendster.pdf}
        }        
    \end{figure}
    \centering
\end{frame}

\begin{frame}{Speedup - Friendster}
    \begin{itemize}
        \item Melhor caso executando com \textit{Hyperthreading} desligado e política \textit{close} - visto também em outras entradas.
        \item Diferente de outras entradas, o caso com \textit{Hyperthreading} liagdo e política \textit{close} tambem se saiu bem. Achamos que pode ser relacionado com a eficiência térmica em grafos maiores.
    \end{itemize}
        
\end{frame}

\begin{frame}{Análise Estatística - Correlação Spearman}
    \begin{figure}
        \centering
        \makebox[\textwidth][c]{%
        \includegraphics[width=0.85\linewidth]{graficos/correlacao_spearman.pdf}
        }        
    \end{figure}
    \centering
        
\end{frame}

\begin{frame}{Análise Estatística - \textit{Feature Importance}}
    \begin{figure}
        \centering
        \makebox[\textwidth][c]{%
        \includegraphics[width=\linewidth]{graficos/feature_importance.pdf}
        }
    \end{figure}
    \centering
    \begin{itemize}
        \item A frequência da CPU é o que mais importa pro \textit{speedup} (estando correlacionada negativamente).
        \item Atrasos na memória afetam mais o \textit{speedup} do que o CPI.
    \end{itemize}
        
\end{frame}

\begin{frame}{Análise Estatística - ANOVA 1}
    Por fim, fizemos uma análise com ANOVA, tentando associar nossas variáveis de entrada com as saídas do \textit{Intel Vtune}

    Assim, resultados com \textit{p-value} menor que 0.05 (5\%) estão marcados em verde e significam que a variável possui um impacto estatisticamente signficativo naquela métrica.
        
\end{frame}

\begin{frame}{Análise Estatística - ANOVA 2}
    \centering
    \resizebox{0.8\textwidth}{!}{%
    \begin{table}[ht]
\centering
\begin{tabular}{lrrl}
\toprule
 & sum\_sq & F & p-value \\
\midrule
C(THREADS) & 21.194140 & 764.778120 & \cellcolor{ForestGreen}{0.000000} \\
C(HT\_OFF) & 0.009423 & 2.720265 & 0.100390 \\
C(BINDING) & 0.040890 & 11.804060 & \cellcolor{ForestGreen}{0.000696} \\
C(HT\_OFF):C(BINDING) & 0.041835 & 12.076720 & \cellcolor{ForestGreen}{0.000606} \\
Residual & 0.831384 & NaN & - \\
\bottomrule
\end{tabular}
\caption{ANOVA para AVERAGE\_CPU\_FREQUENCY (GHz)}
\label{tab:anova_average_cpu_frequency}
\end{table}

    }
    
    ANOVA pra Frequência média da CPU
    
    
    \resizebox{0.8\textwidth}{!}{%
        \begin{table}[ht]
\centering
\begin{tabular}{lrrl}
\toprule
 & sum\_sq & F & p-value \\
\midrule
C(THREADS) & 48.361651 & 18.586491 & \cellcolor{ForestGreen}{0.000000} \\
C(HT\_OFF) & 7.106524 & 21.849600 & \cellcolor{ForestGreen}{0.000005} \\
C(BINDING) & 0.033739 & 0.103734 & 0.747674 \\
C(HT\_OFF):C(BINDING) & 0.294290 & 0.904820 & 0.342450 \\
Residual & 78.059354 & NaN & - \\
\bottomrule
\end{tabular}
\caption{ANOVA para CPI\_RATE}
\label{tab:anova_cpi_rate}
\end{table}

    }
    
    ANOVA pra taxa de ciclos por instrução

        
\end{frame}

\begin{frame}{Análise Estatística - ANOVA 3}
    \centering
    \resizebox{0.8\textwidth}{!}{%
        \begin{table}[ht]
\centering
\begin{tabular}{lrrl}
\toprule
 & sum\_sq & F & p-value \\
\midrule
C(THREADS) & 5300.901527 & 2.754524 & \cellcolor{ForestGreen}{0.006290} \\
C(HT\_OFF) & 3510.528229 & 14.593494 & \cellcolor{ForestGreen}{0.000170} \\
C(BINDING) & 2063.090006 & 8.576399 & \cellcolor{ForestGreen}{0.003733} \\
C(HT\_OFF):C(BINDING) & 672.672057 & 2.796342 & 0.095783 \\
Residual & 57733.038537 & NaN & - \\
\bottomrule
\end{tabular}
\caption{ANOVA para MEMORY\_BOUND}
\label{tab:anova_memory_bound}
\end{table}

    }
    
    ANOVA pra taxa de \textit{Memory Bound}
    
    \resizebox{0.8\textwidth}{!}{%
        \centering
\begin{tabular}{lrrl}
\toprule
 & sum\_sq & F & p-value \\
\midrule
C(THREADS) & 191.385561 & 17.909454 & \cellcolor{ForestGreen}{0.000000} \\
C(HT\_OFF) & 0.508229 & 0.380472 & 0.537936 \\
C(BINDING) & 1.135873 & 0.850341 & 0.357382 \\
C(HT\_OFF):C(BINDING) & 0.734943 & 0.550195 & 0.458963 \\
Residual & 320.588597 & NaN & - \\
\bottomrule
\end{tabular}
\label{tab:anova_speedup}
    }
    
    ANOVA pro \textit{Speedup}

\end{frame}
    
\section{Conclusão}
\begin{frame}{Conclusão do Projeto}
Este trabalho apresentou uma análise detalhada do desempenho do algoritmo PageRank em arquiteturas modernas, explorando o impacto de configurações de multithreading, afinidade de processador
e microarquitetura no tempo de execução.\\
\vspace{0.5em}
Nossa análise demonstrou que para algoritmos limitados por memória como o PageRank a
estratégia padrão de usar o máximo de threads lógicas disponíveis nem sempre resulta no melhor
desempenho.
\end{frame}



\section{Conclusão}
\begin{frame}{Recomendações e possíveis melhorias}
Tivemos algumas ideias para melhorar o desempenho da nossa aplicação, como:
 \begin{enumerate}
     \item Reduzir o tempo gasto por atrasos na memória. Isso poderia ser feito utilizando um processador com barramento maior, ou com memória \textit{cache} maior.
     \item Utilizar uma máquina mais atual que a blaise. O processador dela foi lançado a 9 anos atrás, e tecnologias mais recentes (ex. \textit{3D VCache} da AMD) podem melhorar o desempenho dessa aplicação.
 \end{enumerate}

\end{frame}

\section{Conclusão}
\begin{frame}{Dificuldades}
 \begin{itemize}
     \item Em  diversos momentos a máquina utilizada (blaise) estava alocada por outra pessoa.
     \item Organizar os dados gerados pelo Intel Vtune Profiler. Ele gera muitos dados, e a maioria não
era de interesse do nosso grupo. 
     \item Google Colab não apresentou um \textit{pair programming} satisfatório.
 \end{itemize}    
\end{frame}

\section{Conclusão}
\begin{frame}{Aprendizados}
 \begin{itemize}
     \item A importâncida da aplicação de um projeto experimental sólido e replicável.
     \item Utilização de análise estatística criteriosa para verificar correlações entre os parâmetros do experimento.     
 \end{itemize}    
\end{frame}

\section*{}
\begin{frame}
    \frametitle{Obrigado!}
    \InfContacts    
    \vspace{1em}
    \centering
    \tiny % Deixa a fonte bem pequena
    Todos dados disponíveis no repositório: \url{https://github.com/thgdsg/perf-analysis}
\end{frame}

\end{document}